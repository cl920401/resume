%% start of file `resume.tex'.
%% Copyright 2006-2013 Xavier Danaux (xdanaux@gmail.com).
%
% This work may be distributed and/or modified under the
% conditions of the LaTeX Project Public License version 1.3c,
% available at http://www.latex-project.org/lppl/.

\documentclass[12pt,a4paper,sans]{moderncv}   % possible options include font size ('10pt', '11pt' and '12pt'), paper size ('a4paper', 'letterpaper', 'a5paper', 'legalpaper', 'executivepaper' and 'landscape') and font family ('sans' and 'roman')

% moderncv 主题
\moderncvstyle{classic}                       % 选项参数是 ‘casual’, ‘classic’, ‘oldstyle’ 和 ’banking’
\moderncvcolor{blue}                          % 选项参数是 ‘blue’ (默认)、‘orange’、‘green’、‘red’、‘purple’ 和 ‘grey’
%\nopagenumbers{}                             % 消除注释以取消自动页码生成功能

%windows 字符编码配置
%\usepackage[utf8]{inputenc}                   % 替换你正在使用的编码
%\usepackage{CJKutf8}
% 字符编码
\usepackage[UTF8, heading = false, scheme = plain]{ctex}
\setCJKmainfont{Hiragino Sans GB}

% 调整页面出血
\usepackage[scale=0.9]{geometry}
\setlength{\hintscolumnwidth}{3cm}           % 如果你希望改变日期栏的宽度
\linespread{1.5}

% 个人信息
\name{个人简历}{陈亮}
\title{高级后端研发工程师}                     % 可选项、如不需要可删除本行
\address{武汉}{江岸区}            % 可选项、如不需要可删除本行
\photo[64pt][0.1pt]{avatar}                  % ‘64pt’是图片必须压缩至的高度、‘0.4pt‘是图片边框的宽度 (如不需要可调节至0pt)、’picture‘ 是图片文件的名字;可选项、如不需要可删除本行

% 显示索引号;仅用于在简历中使用了引言
%\makeatletter
%\renewcommand*{\bibliographyitemlabel}{\@biblabel{\arabic{enumiv}}}
%\makeatother

% 分类索引
%\usepackage{multibib}
%\newcites{book,misc}{{Books},{Others}}
%----------------------------------------------------------------------------------
%            内容
%----------------------------------------------------------------------------------
\begin{document}
%\begin{CJK}{UTF8}{gbsn}                       % windows配置
\maketitle{\vspace*{-2.5em}}

\section{个人信息}
\cvdoubleitem{手机/微信}{~189~8612~0001}{邮箱}{cl920401@163.com}
\cvdoubleitem{link}{https://github.com/cl920401}{QQ}{408194766}
\cvitem{工作年限}{6年}

\section{教育背景}
\cvitem{2010.6--2014.6}{江汉大学--自动化/武汉理工大学--电子商务~本科(双学位)}

\section{工作技能}
\cvitem{英语水平}{CET-4,可以阅读英文文档和英文材料,能够阅读和撰写英文邮件}
\cvitem{编程技能}{%
\begin{itemize}%
    \item 开发语言:(按了解程度排序)golang、c++、java、x86汇编、arm汇编、python、js
    \item 开发平台:linux、windows、web、android
    \item 开发工具:nginx、mysql、redis、mongoDB、kafka、etcd、k8s、docker、shell、git等
\end{itemize}}
\cvitem{个人荣誉}{《摇一摇技术方案设计》获得网约车技术部-月度优秀文档第一名}

\section{工作经历}
\cventry{2021.03--2021.09}{滴滴出行}{乘客运营}{高级Golang后端研发}{D6}{}
\cvitem{}{负责裂变中台服务开发,支持公司的网约车、橙心优选等相关业务的裂变增长类营销活动页的上线迭代,并设计了架构接入层部分,以支持各业务线快速接入,提升人效}
\cventry{2017.12--2021.03}{猎豹移动}{AI工程平台}{高级Golang后端研发}{T4}{}
\cvitem{}{负责过机器人、工具app的后端接口服务和Web后台和组件的后端核心研发工作,目前负责机器人miniToC业务线的后端接口人}
\cventry{2016.7--2017.12}{猎豹移动}{游戏业务部}{高级Windows C++客户端研发}{T4}{}
\cvitem{}{主要负责金山毒霸-游戏盒子产品迭代的开发和bug修复工作,独立跟进业务需求以及设计并重构优化框架代码}
\newpage
\cventry{2015.5--2016.7}{四象联创}{PC组}{Windows 浏览器研发}{}{}
\cvitem{}{主要负责支持PC浏览器产品迭代的开发和bug修复工作,独立完成需求研发,对chrome内核源码做定制化的二次开发}

\section{项目经验}

\subsection{\textbf{滴滴出行 - 乘客运营 - 裂变中台服务端开发}}
\cventry{}{滴滴社交运营 - 裂变中台}{}{}{}{}
\cvitem{项目介绍}{裂变中台是滴滴出行的裂变增长类营销活动的中台服务,承接网约车、橙心优选、滴滴货运、出租车等业务方的用户增长裂变需求}
\cvitem{工作内容}{前期设计了架构接入层模块Cyberset,根据现有的增长营销活动抽象出统一的流程,制定整体接入层接入规范,设计代码多态来支持不同业务线的个性化需求,提升整体开发人效;后期主要支持各业务线的需求上线迭代}
\cvitem{技术栈}{golang、linux、k8s、etcd、docker、elk、mysql、redis、kafka、git}

\subsection{\textbf{猎户星空 - AI工程平台 - 机器人后端及数据中台服务}}
\cventry{}{猎户AI服务机器人 - Mini 服务端接口人}{}{}{}{}
\cvitem{项目介绍}{mini机器人是ToC的机器人,参与过公司服贸会展出,需要对接机器人端系统层到应用层,官方app,nlp平台等,是一个链路超长、比较庞大的系统}
\cvitem{工作内容}{服务端组owner(平均3~5人),负责超过5个部门的跨部门对接,结合ci工具和测试驱动对开发流程优化,提升开发效率;整体微服务架构的设计,基于kong网关插件实现了非侵入式的业务账号鉴权}
\cvitem{技术栈}{golang、linux、k8s、etcd、docker、kong、elk、mysql、redis、kafka、mqtt、git}

\cventry{}{Orionbase、DMP - Golang数据中台及组件研发}{}{}{}{}
\cvitem{项目介绍}{机器人开放平台OrionBase是一个机器人二次开发平台,主要支持对机器人的远程发布和操作,并且提供nlp配置,查看崩溃,数据总览,监控预警等基础服务。DMP平台是海外工具的数据中台,负责海外App的资源配置,广告投放,数据挖掘,数据监控等数据业务}
\cvitem{工作内容}{主要负责权限和账号体系的设计和研发,实现了灵活配置的账号权限管理;并且重构了mq消费队列,两阶段提交保证数据一致性,并且定义通用数据协议,最大程度复用;协同运维推进公司的k8s自动化部署流程,优化了分支管理,提升多人协同开发效率;通过facebook广告数据接口优化,接口响应时间缩减到1/10;平台整体操作日志的设计和研发}
\cvitem{技术栈}{golang、linux、k8s、docker、kong、elk、mysql、redis、Elasticsearch、kafka、git}

\subsection{\textbf{猎豹移动 - 海外工具平台 - 海外APP后端}}
\cventry{}{海外工具APP - 后端研发}{}{}{}{}
\cvitem{项目介绍}{海外各种APP的后端服务,高并发场景下的开发、维护和迭代。}
\cvitem{工作内容}{部分老项目的接手需要在文档缺失严重的情况下进行维护,新项目支持业务需求迭代,提供RestfulAPI,通过严格的压测流程适应各种高并发场景,保障服务的可用性99.99}
\cvitem{技术栈}{golang、python、c++、java、php、linux、nginx、mysql、redis、mongodb、git}

\cventry{}{icfun - Java游戏后端研发}{}{}{}{}
\cvitem{项目介绍}{icfun是一款安卓小游戏集合软件,googleplay已上线,搜索Icfun可以体验,国内版小米商店搜索趣游戏}
\cvitem{工作内容}{从框架设计到对接客户端到开发上线包括后期的线上修复,全流程参与,从0到1设计游戏匹配服务(人人对战,人机对战),设计实现了可自定义的匹配策略,并且规范了压测流程,其他组follow}
\cvitem{技术栈}{java (spring、netty)、linux、nginx、mysql、redis、falcon、git}

% \newpage
\subsection{\textbf{猎豹移动 - 金山毒霸 - 猎豹游戏盒子}}
\cventry{}{游戏盒子 - C++ Windows研发}{}{}{}{}
\cvitem{项目介绍}{一款页游平台win客户端,支持游戏加速,游戏辅助等}
\cvitem{工作内容}{维护金山游戏盒子,且独立完成金山游戏盒子的重写,从0到1完成开发上线,完成登录注册,小号管理,截图,视野缩放,游戏加速,游戏挂机(敏感问题,此功能没有上线)等功能,项目到后期出于兴趣开始自学go语言,接一些简单的后端需求,分担后端同学的压力}
\cvitem{软件地址}{http://box.wan.liebao.cn/website/}
\cvitem{技术栈}{c++、js、golang、windows、SOUI、IE内核、svn}

% \newpage
\subsection{\textbf{hao123 - 桔子浏览器}}
\cventry{}{C++ 浏览器内核研发}{}{}{}{}
\cvitem{项目介绍}{当时是hao123官方浏览器,百度合作项目,基于IE和chrome内核,主打安装包体积小,安装包不到2M}
\cvitem{工作内容}{参与产品的开发迭代,并且修复现有bug,参与的开发任务主要有双核版的预研和开发、自动化构建打包平台和截图、视频下载、图片放大器等特色功能点。其中最大的收获是学习了部分chrome源码,了解到开源的魅力}
\cvitem{技术栈}{c++、js、php、python、windows、WTL、IE内核、chrome内核、svn}

\newpage
\section{自我评价}
\cvitem{}{个人主观的判断自己有如下特质:}
\cvitem{}{1、较扎实的C++的基础,让我对底层原理部分有更好的理解能力,能快速学习,对技术深度有一定要求}
\cvitem{}{2、总结个人的主要发展过程基本是兴趣驱动,对于新的技术知识有强烈的学习热情}
\cvitem{}{3、注重协作效率,有一些带团队的经验,整体综合素质相较于纯技术向的研发更强,视野相对更全面,同时个人也非常有推进团队的技术发展的意愿,注重分享和可持续发展}
\cvitem{}{4、较多的从0到1,1到100的经验,从头到尾跟进一个项目的过程中积累一些避坑意识,往往在前期方案设计上愿意花更多的时间,以减轻后期维护成本,不轻易欠下技术债}
\cvitem{}{如果贵公司正好也认可这些个人的专业素质,我相信合作会很愉快\^{}\_\^{}}



% 来自BibTeX文件但不使用multibib包的出版物
%\renewcommand*{\bibliographyitemlabel}{\@biblabel{\arabic{enumiv}}}% BibTeX的数字标签
              % 'publications' 是BibTeX文件的文件名

% 来自BibTeX文件并使用multibib包的出版物
%\section{出版物}
%\nocitebook{book1,book2}
%\bibliographystylebook{plain}
%\bibliographybook{publications}               % 'publications' 是BibTeX文件的文件名
%\nocitemisc{misc1,misc2,misc3}
%\bibliographystylemisc{plain}
%\bibliographymisc{publications}               % 'publications' 是BibTeX文件的文件名

\clearpage%\end{CJK}  %windows配置
\end{document}

%% 文件结尾 `resume.tex'.
