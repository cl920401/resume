%% start of file `resume.tex'.
%% Copyright 2006-2013 Xavier Danaux (xdanaux@gmail.com).
%
% This work may be distributed and/or modified under the
% conditions of the LaTeX Project Public License version 1.3c,
% available at http://www.latex-project.org/lppl/.

\documentclass[12pt,a4paper,sans]{moderncv}   % possible options include font size ('10pt', '11pt' and '12pt'), paper size ('a4paper', 'letterpaper', 'a5paper', 'legalpaper', 'executivepaper' and 'landscape') and font family ('sans' and 'roman')

% moderncv 主题
\moderncvstyle{classic}                       % 选项参数是 ‘casual’, ‘classic’, ‘oldstyle’ 和 ’banking’
\moderncvcolor{blue}                          % 选项参数是 ‘blue’ (默认)、‘orange’、‘green’、‘red’、‘purple’ 和 ‘grey’
%\nopagenumbers{}                             % 消除注释以取消自动页码生成功能

%windows 字符编码配置
%\usepackage[utf8]{inputenc}                   % 替换你正在使用的编码
%\usepackage{CJKutf8}
% 字符编码
\usepackage[UTF8, heading = false, scheme = plain]{ctex}
\setCJKmainfont{Hiragino Sans GB}

% 调整页面出血
\usepackage[scale=0.9]{geometry}
\setlength{\hintscolumnwidth}{3cm}           % 如果你希望改变日期栏的宽度
\linespread{1.5}

% 个人信息
\name{个人简历}{陈亮}
\title{高级后端研发工程师}                     % 可选项、如不需要可删除本行
\address{北京}{朝阳区}            % 可选项、如不需要可删除本行
\photo[64pt][0.1pt]{avatar}                  % ‘64pt’是图片必须压缩至的高度、‘0.4pt‘是图片边框的宽度 (如不需要可调节至0pt)、’picture‘ 是图片文件的名字;可选项、如不需要可删除本行

% 显示索引号;仅用于在简历中使用了引言
%\makeatletter
%\renewcommand*{\bibliographyitemlabel}{\@biblabel{\arabic{enumiv}}}
%\makeatother

% 分类索引
%\usepackage{multibib}
%\newcites{book,misc}{{Books},{Others}}
%----------------------------------------------------------------------------------
%            内容
%----------------------------------------------------------------------------------
\begin{document}
%\begin{CJK}{UTF8}{gbsn}                       % windows配置
\maketitle{\vspace*{-2.5em}}

\section{个人信息}
\cvdoubleitem{手机/微信}{+86~189~8612~0001}{邮箱}{cl920401@163.com}
\cvdoubleitem{link}{https://github.com/cl920401}{QQ}{408194766}
\cvitem{工作年限}{4年}

\section{教育背景}
\cvitem{2010.6--2014.6}{江汉大学--自动化/武汉理工大学--电子商务~本科(双学位)}
\cvitem{2013.5--2015.5}{自学C++、windows内核、x86汇编逆向等技术}

\section{工作技能}
\cvitem{英语水平}{\emph{CET-4,可以阅读英文文档和英文材料,能够阅读和撰写英文邮件。} }
\cvitem{编程技能}{%
\begin{itemize}%
    \item 开发语言:(按了解程度排序)c++、golang、java、x86汇编、arm汇编、python、js
    \item 开发平台:linux、window、android
    \item 开发技能:nginx、mysql、redis、mongoDB、shell、git、svn等
\end{itemize}}
{\vspace*{-1.5em}}
\section{项目经验}

\subsection{\textbf{DMP平台}}
\cventry{}{Golang后端研发工程师}{}{}{}{}
\cvitem{项目介绍}{DMP平台是提供给部门内部人员使用的管理后台,接入了部门所有的后台页面,为各个业务线提供统一的账号体系、邮件提醒、权限配置等基础功能。}
\cvitem{工作内容}{负责大部分权限管理相关接口,有时也需要协助对接的业务后台开发,如帮助广告核算平台接入facebook广告成效分析接口等。}
\cvitem{技术栈}{golang、linux、mysql、redis、git}

\subsection{\textbf{icfun、趣游戏}}
\cventry{}{Java后端研发工程师}{}{}{}{}
\cvitem{项目介绍}{一款安卓小游戏集合软件,googleplay已上线,搜索Icfun可以体验,国内版小米商店搜索趣游戏。}
\cvitem{工作内容}{从框架设计到对接客户端到开发上线包括后期的线上修复,全流程参与,承担了整个团队80\%的工作量,完成了账号体系,盒子主页和游戏列表的配置下发,以及在线自动匹配和手动开房间进行游戏(websocket)等。由于产品策略,国内版不支持多人对战。}
\cvitem{技术栈}{java (spring、netty)、linux、nginx、mysql、redis、falcon、git}

\subsection{\textbf{清理云服务}}
\cventry{}{后端研发工程师}{}{}{}{}
\cvitem{项目介绍}{清理云控制猎豹清理大师的清理规则,老的核心业务,数据量较大,但是开发思路陈旧,坑也很多。}
\cvitem{工作内容}{文档缺失严重的情况下,维护上百台机器正常运行,根据已知源码分析解决和跟进反馈的问题。}
\cvitem{技术栈}{golang、python、c++、java、php、linux、nginx、mysql、redis、mongodb、git}

\subsection{\textbf{区块链相关}}
\cventry{}{后端研发工程师}{}{}{}{}
\cvitem{项目介绍}{18年公司对区块链做了很多尝试,其中参与了很多很杂的项目,包括去中心化交易所的实时价格抓取和涨跌推送,Dapp后端(社交聊天模块),Eth自建节点的优化等}
\cvitem{工作内容}{区块链相关的项目比较杂,但也打开了眼界,学到很多以前接触不到的技术点。其中主要涉及到的技术点有,安卓ios的推送原理,signal聊天加密协议源码阅读,eth交易过程源码阅读、自学前端vue.js完成后台页面、docker服务的打包和分布式部署等。}
\cvitem{技术栈}{python、java、go、vue.js、linux、android、docker、nginx、mongodb、PostgreSQL、redis、mysql、git}
%\newpage
\subsection{\textbf{猎豹游戏盒子}}
\cventry{}{C++ windows客户端研发工程师}{}{}{}{}
\cvitem{项目介绍}{一款页游平台win客户端,支持游戏加速,游戏辅助等。}
\cvitem{工作内容}{独立从0到1完成开发上线,完成登录注册,小号管理,截图,视野缩放,游戏加速,游戏挂机(敏感问题,此功能没有上线)等功能,项目到后期开始自学go语言,完成一些简单的后端业务接口,为转岗服务端奠定了基础。另外同时也在维护一款内置在金山毒霸的金山游戏盒子,因为项目老,需求较少,在简历中省略。}
\cvitem{技术栈}{c++、js、golang、windows、SOUI、IE内核、svn}

\newpage
\subsection{\textbf{hao123桔子浏览器}}
\cventry{}{C++ windows客户端研发工程师}{}{}{}{}
\cvitem{项目介绍}{当时是hao123官方浏览器,基于IE内核,主打安装包体积小,安装包不到2M。}
\cvitem{工作内容}{参与产品的开发迭代,并且修复现有bug,参与的开发任务主要有双核版的预研和开发、自动化构建打包平台和截图、视频下载、图片放大器等特色功能点。其中最大的收获是学习了部分chrome源码,了解到开源的魅力。}
\cvitem{技术栈}{c++、js、php、python、windows、WTL、IE内核、chrome内核、svn}

\section{工作经历}
\cventry{2017.12----至今}{猎豹移动}{高级Golang后端研发}{T4}{}{}
\cvitem{}{后端接口开发、服务器维护、技术架构、devops、技术分享等}
\cvitem{}{负责工具后端组的需求预研、开发和维护等,同时也负责游戏后端组的后端开发、框架搭建、devops建设等工作}
\cventry{2016.7--2017.12}{猎豹移动}{高级C++客户端研发}{T4}{}{}
\cvitem{}{客户端研发、产品迭代跟进、技术文档等}
\cvitem{}{主要负责windows平台两个游戏盒子产品迭代的开发和bug修复工作}
\cventry{2015.5----2016.7}{四象联创}{C++客户端研发}{}{}{}
\cvitem{}{客户端研发、团队技术交接等}
\cvitem{}{主要负责PC浏览器产品迭代的开发和bug修复工作}
\cventry{}{}{}{}{}{}

\section{自我评价}
\cvitem{}{能熟练运用c++、java、golang、python、汇编等语言解决相对复杂的需求,其中对c++和golang有浓厚的兴趣,理解会更为深入一些。对于nginx、redis、mysql、git、docker等常用的工具和数据库也有自己的见解。有一定的架构思维,了解devop,能从0开始搭建后端开发框架。开发思路清晰,学习能力强,对于新接触的框架和语言都能快速上手,并且善于理解实现原理,避开部分坑。能迅速响应需求拥抱变化,高效率与合作部门沟通。另外有两年的windows桌面浏览器开发经验,熟悉chrome内核源码,有内核定制能力。工作前也自学了windows内核和汇编等底层技术。总的来说知识体系比较全面,学习能力很强。}

% 来自BibTeX文件但不使用multibib包的出版物
%\renewcommand*{\bibliographyitemlabel}{\@biblabel{\arabic{enumiv}}}% BibTeX的数字标签
              % 'publications' 是BibTeX文件的文件名

% 来自BibTeX文件并使用multibib包的出版物
%\section{出版物}
%\nocitebook{book1,book2}
%\bibliographystylebook{plain}
%\bibliographybook{publications}               % 'publications' 是BibTeX文件的文件名
%\nocitemisc{misc1,misc2,misc3}
%\bibliographystylemisc{plain}
%\bibliographymisc{publications}               % 'publications' 是BibTeX文件的文件名

\clearpage%\end{CJK}  %windows配置
\end{document}

%% 文件结尾 `resume.tex'.
